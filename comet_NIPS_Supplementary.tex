\documentclass{article} % For LaTeX2e
\usepackage{nips15submit_e,times}
\usepackage{hyperref}
\usepackage{url}
%\documentstyle[nips14submit_09,times,art10]{article} % For LaTeX 2.09
%% COMET packages
\usepackage{amssymb}
\usepackage{amsmath}
\usepackage{mathtools}
\usepackage{hyperref}
\usepackage[square,numbers]{natbib}
\bibliographystyle{unsrtnat}
\usepackage{caption}

%% cross reference the main document as N-
\usepackage{xr}
\externaldocument{comet_main_NIPS}
%\usepackage{xcite} %cross-documents-bibliography.
%\externalcitedocument{comet_main_NIPS} %cross-documents-bibliography

%%%%%%%%%%%%%%%%%%%%%%%%
%% packeges from ICML
% use Times
\usepackage{times}
\usepackage{graphicx} % more modern
\usepackage{subfigure} 
\usepackage{amsthm}
\usepackage{natbib}
\usepackage{algorithm}
\usepackage{algorithmic}
\usepackage{hyperref}
%\newcommand{\theHalgorithm}{\arabic{algorithm}}

%%%% COMET commands %%%%
\newcommand\todo[1]{\textbf{<ToDo:#1}!}
%\newcommand\mat[1]{\mathcal{#1}}
%\newcommand\mat[1]{\boldmath{#1}}
\newcommand\mat[1]{{#1}}
\renewcommand\vec[1]{\mathbf{#1}}
\newcommand{\T}{{}^\mathsf{T}}
\newcommand{\W}{\mat{W}}
\newcommand{\E}{\mat{E}}
\newcommand{\Hh}{\mat{H}}
\newcommand{\Pp}{\mat{P}}
\newcommand{\newW}{{\mat{W^{new}}}}
\newcommand{\eqdef}{\doteq}
\newcommand{\Rd}{\mathbb{R}^d}
\newcommand{\R}{\mathbb{R}}
\newcommand{\tL}{\tilde{L}(\W)}
\newcommand{\frobsq}[1]{{\|#1\|_F^2}}
\newcommand{\frob}[1]{{\|#1\|_F}} 
\newcommand{\ignore}[1]{}

\newcommand{\q}{{\vec{q}}}
\newcommand{\p}{{\vec{p}}}
\newcommand{\trip}{{t}}
\newcommand{\qt}{{\q_{\trip}}}
\newcommand{\pt}{{\p_{\trip}}}
\newcommand{\triplet}{(\qt, \pt^{+}, \pt^{-})}


\newcommand{\A}{A}
\newcommand{\B}{\vec{b}}
\newcommand{\C}{c}
\newcommand{\invA}{A^{-1}}

\newcommand{\grd}{\frac{\partial \tL}{\W}}
\newcommand{\grdkl}{\frac{\partial \tL}{\W_{kl}}}


\newcommand{\uscalar}{{u}_{1}}
\newcommand{\uvec}{\vec{u}_{2:d}} 
\newcommand{\Wvec}{\W_{2:d,1}}
\newcommand{\Wscalar}{\W_{1,1}}


\newtheorem{theorem}{Theorem}
\newtheorem{lemma}{Lemma}
\newtheorem{corollary}{Corollary}
\newtheorem{definition}{Definition}
\newtheorem{apptheorem}{Theorem}
\newtheorem{applemma}{Lemma}

\renewcommand{\eqref}[1]{Eq.~(\ref{#1})}
\newcommand{\figref}[1]{Fig.~\ref{#1}}
\newcommand{\secref}[1]{Sec.~\ref{#1}}
\newcommand{\tabref}[1]{Table~\ref{#1}}

%\DeclareMathOperator*{\argmin}{arg\,min}


%%%%%%%%%%%%%%%%%%%%%%%%%%%%%%%%%

\begin{document}

\section*{Supplementary Material: Metric Learning One Feature at a Time}

% ==============================================================
\appendix
\section*{Appendix A: Details of gradient derivation for a triplet}
\label{appendix-grad}

To compute matrix gradient step $\frac{\partial {l_t (\W)}}{\partial \W}$ of an arbitrary triplet $t$, we denote the linear part of the hinge loss of a triplet $t$ by $\lambda_{W}^t \eqdef 
1-\qt\T \W \pt^{+} + \qt\T\W\pt^{-}.$

$\W$ is PD and therefore symmetric. We enforce its gradient to be symmetric by replacing $\W$ with $\tfrac{1}{2}(\W + \W\T)$.
The derivative of the ranking loss is then given by
\begin{equation}
\frac{\partial {l_{\W}^{t}}}{\partial \W} = \tfrac{1}{2}[\vec{q}_{t}\Delta\vec{p}_{t}\T  + \Delta\vec{p}_{t}\vec{q}_{t}\T]\cdot {l'}(\lambda_{W}^t)
\label{dlossranking}
\nonumber 
\end{equation} where $l'(x) \eqdef \frac{d{l(x)}}{dx}$ is the outer derivative of the loss function, $\Delta\vec{p}_{t} \eqdef (\vec{p}_{t}^{-} - \vec{p}_{t}^{+})$.

\section*{Appendix B: Updating the inverse matrices}
\label{appendix-inverse}

To compute efficiently the updates of \eqref{PDUpdateCondQuadForm} we update $\W^{-1}$ following a block coordinate step, and derive $\invA$ from $\W^{-1}$ before the next step. Both terms take $O(d^2)$ to compute.

$\newW^{-1}$ can be easily computed using the Woodbury matrix
identity \cite{woodbury1950inverting}. We rewrite \eqref{updateEq} and \eqref{gradMtx} (of the main paper), using $\newW = \W + \eta G = \W+\mat{\widetilde{G}}$
 %\label{updateEqWDB}
and write
\begin{equation}
  \mat{\widetilde{G}} = \mat{U}\mat{C}\mat{V} = \left[ \begin{matrix}
      \vec{u} & \vec{e_k} \end{matrix} \right] \left[ \begin{matrix}
      \eta & 0 \\ 0 & \eta \end{matrix} \right] \left[ \begin{matrix}
      \vec{e_k}\T \\ \vec{u}\T \end{matrix} \right],
  \label{gradMtxWDB}
  \nonumber 
\end{equation}
where $\vec{u}$ is a column vector that equals the column $k$ of the gradient matrix of the objective \eqref{gradMtx} (on the main paper) ,
$\vec{e_k}$ equals an elementary vector for selecting a column $k$ of
a matrix. 
Using the Woodbury matrix identity gives 
\begin{equation}
    \begin{array}{lcl}
    \newW^{-1} = 
    \W^{-1} - \W^{-1} \mat{U} (\eta^{-1} I_2 + \mat{V}     \W^{-1} \mat{U})^{-1} \mat{V} \W^{-1}
    \end{array}
    \nonumber
    \label{InvWwdb}
\end{equation}

Last, we evaluate $\invA$ before a coordinate step given $\W$
and $\W^{-1}$, using the Schur complement and its corresponding
notation \eqref{schurNotationPreUpdate} (on the main paper):
\begin{equation}
\begin{array}{l}
 \W^{-1} \!\!=\!\! 
 \left[ \begin{array}{cc} s & -s \B\T \invA \\ -s \B\T \invA\T &  \invA \!+ \!\invA \B s \B\T \invA  \end{array}  \right]
\end{array}
\label{BlockInvW}
\end{equation} 
where $s= \C-\B\T \invA \B$ is a scalar denoting the Schur Complement. This gives us four terms: (1) $s = \W^{-1}_{1,1}$, (2)
$  -s \B\T \invA = -\W^{-1}_{1,1} \B\T \invA = \W^{-1}_{1,2:d}$, (3) $\B\T \invA = -\frac{\W^{-1}_{1,2:d}}{\W^{-1}_{1,1} }$ and (4) $\invA\B = (\B\T \invA)\T$. Subtituting them in the lower right block of \eqref{BlockInvW} yields $\invA + \frac{1}{\W^{-1}_{1,1} } \W^{-1}_{2:d,1} (\W^{-1}_{2:d,1})\T = \W^{-1}_{2:d,2:d}$. Rearranging the last term gives
\begin{equation}
  \invA = \W^{-1}_{2:d,2:d}- \frac{\W^{-1}_{2:d,1} {\W^{-1}_{1, 2:d}}^{\T}}{\W^{-1}_{1,1}}. 
  \label{InvA}
\end{equation}
Computing \eqref{InvA} has a complexity of $O(d^2)$.

\section*{Appendix C: Analysis of computational complexity}

We first evaluate the computational complexity of a single coordinate step \eqref{gradMtx}, including the computation of the gradient and updating of $\W$, $\W^{-1}$ and $\invA$. Consider first the computation of the gradient. For the hinge-loss case $l^{h}_W$, each element $\delta_{i,j}$ of the gradient matrix \eqref{gradMtx} equals
\begin{equation}
    \delta_{(i,j)} = \sum\limits_{t\in \cal{T}}{ [\tfrac{1}{2}[(\vec{q}_{t})_i(\Delta\vec{p}_{t}\T)_j + (\Delta\vec{p}_{t}\T)_i(\vec{q}_{t})_j\T] } \cdot \textbf{1}(\lambda_{W}^t) - \alpha \cdot \W^{-1}_{i,j} + \beta \cdot \W_{i,j},
\label{gradMatElem}
\end{equation}
where $\lambda_{W}^t \eqdef 1+\qt\T \W \Delta\p_{t}$ is the linear part a triplet loss

For dense data, evaluating the sum over $T$ triplets costs $O(T)$ operations. However, for $gamma$-sparse data with a sparsity coefficient $ 0< \gamma <1 $, evaluating the sum in \eqref{gradMatElem} costs an average of $O(\gamma^2 T)$ operations, because we can accumulate only the elements that are both non-zeros in $(\vec{q}_{t})_i$ and in $(\Delta\vec{p}_{t}\T)_j  $ and likewise for $(\vec{q}_{t})_j$ and $(\Delta\vec{p}_{t}\T)_i$.   To efficiently evaluate the indicator functions $\{ \textbf{1}(\lambda_{W}^t) \}_{t \in T}$ on \eqref{gradMatElem}, we keep an array of the linear terms $\{\lambda_{W}^t\}_{t \in T}$. Computing all the gradient elements $\delta_{(k,1:d)}$ in a single row $k$ costs $O(d\cdot \gamma^2 T)$.
Maintaining and updating $\W^{-1}$ and $\invA$, and computing the optimal step size following equations on Appendix B (in supplementary), each costs $O(d^2)$ operations. 
To conclude, the total computational complexity per block-coordinate step is $O(\gamma^2 d T + d^2)$. When taking $Nd$ coordinate steps, the overall complexity of COMET is 
\begin{equation}
O(N \cdot (\gamma d)^2 T + N \cdot d^3)
\label{cometComplexity}
\end{equation}
We found empirically that COMET converges within $N= 5 - 10$. As a comparison, consider using SGD or mini-batches for the objective of \eqref{hingelt} and projecting onto the PD cone every $P$ triplets ($P \ll T$), as proposed in \cite{OASIS,qian}. The computational complexity per data pass becomes $O((\gamma d)^2 T + \frac{T}{P} d^3)$. This approach is slower than COMET and only reaches the complexity of COMET when projections are very rare. For example, \citet{qian} used mini-batches of $P=10$ triplets. With a total of $T=100k$ triplets, yielding a computational complexity ~1000 times larger than COMET.

Compare further with \citet{hdsl}. COMET’s complexity is better than HDSL. The fast heuristic version of HDSL is $O(Md+Tk)$ per coordinate step, where $M$ is the size of mini-batch, $k$ is the iteration number. This is summed over $k=1,...,O(d^2)$ iterations, since each HDSL step considers a single pair of features, updating 4 matrix entries as opposed to $2d-1$ entries in COMET. Overall, this yields $O(Md^3+Td^4)$ computations for HDSL, compared with \eqref{cometComplexity} of COMET. Since both $M$ and $N$ are typically small, this means HDSL is more costly than COMET by a factor of $\frac{d^2}{N}$. In our experiments, HDSL sometimes uses much less than $O(d^2)$ iterations, but then achieves a significantly inferior test error.

Finally, we compare with the complexity of LEGO \cite{lego}. LEGO requires $O(d^2)$ computation per constraint. Thus for $N$ passes over $T$ triplets LEGO's complexity is $O(N\cdot d^2 \cdot T)$. Assuming an equal number of passes over the triplet, we find that COMET's complexity \eqref{cometComplexity} is asymptotically better than LEGO as long as $N \cdot T \cdot (1-\gamma^2) > d^2$. That is, whenever the data is even moderately sparse, and the number of triplets is larger than the number of matrix parameters.

{\bf COMET Memory footprint}: Keeping the data triplets in memory takes $O(\gamma d T)$ elements and holding $\W$ and $\W^{-1}$ costs $O(d^2)$. The total memory usage is $O(\gamma d T + d^2)$. 

\section*{Appendix D: Proofs and Lemmas}
\label{appendix-proofs}
\begin{lemma}[Smooth objective]
\label{lem:smooth}

Let:

$\tL=\sum\limits_{t\in T}{l_{\W + \kappa I}(\vec{q}_t, \vec{p}_{t}^{+}, \vec{p}_{t}^{-})} -
\alpha \cdot \log \det(\W + \kappa I) + \tfrac{\beta}{2}  \cdot \| \W + \kappa I \|_{F}^{2}$, 
where $l_{\W + \kappa I}$ is either the squared hinge loss or the log-loss, and $\tL$ is defined over the positive semidefininte cone. 
Let $\Hh^i \in \R^{d \times d}$, $i=1 \ldots d$, be a symmetric matrix with non-zero entries only on the $i$-th row and column.
For any $\W$ and $\Hh^i$ such that $\W + \Hh^i$ is PSD, there exists a positive constant $M_i$ such that:
\begin{equation}
\label{eq:ineq}
\Delta L \leq  \langle \grd, \Hh^i \rangle + \frac{M_i}{2} \frobsq{\Hh^i} = \sum_{k,l=1}^d  \grdkl \Hh_{kl}^i + \frac{M_i}{2} \sum_{k,l=1}^d  (\Hh_{kl}^i)^2, \nonumber
\end{equation}
with the constant $M_i \leq  2 \sum_{t=1}^T (\qt_i^2 +{\Delta\vec{p}_{t}}_i^2) + \frac{\alpha d}{\kappa ^2} + \beta$ and $\Delta L = \tilde{L}(\W + \Hh^i) - \tL$.
\end{lemma}
\begin{proof}%[\bf{Proof of Lemma 1}]
The objective $\tL$ is comprised of three terms: (1) the sum of loss terms, (2) the $\log \det$ term, and (3) the Frobenius regularization term. We will bound each of the separately, denoting the positive bounding constants $M^1_i$, $M^2_i$ and $M^3_i$, respectively. 
%The Frobenius norm term ensures that $\tL$ is at least $\beta$ strongly-convex.

Assuming the instances $\qt$ and $\pt$ are unit normalized, straightforward computation shows that for the term (1), inequality \ref{eq:ineq} holds true for $M^1_i \leq 2 \sum_{t=1}^T (\qt_i^2 +{\Delta\vec{p}_{t}}_i^2)$. %This means that if the features are more or less equally weighted, $M^1_i$ is very roughly on the order of $\frac{T}{d}$.

To show that \ref{eq:ineq} is true for the $-\log \det$ term, we bound the maximal eigenvalue of its Hessian $\mathcal{H}$, which upper bounds $M_i^2$ by convexity and standard use of a Taylor expansion.
The Hessian is a $d^2 \times d^2$ PSD matrix, due to convexity and twice-differentiability of $- \log \det$. At every point $\mat{X} = \W + \kappa I$, $\W \succ 0$, the Hessian $\mathcal{H}(\mat{X})$ defines a bilinear form $\mathcal{B}_{\mat{X}}\left(\mat{P},\mat{Q}\right)$ on the set of symmetric $d \times d$ matrices. This bilinear form is $\mathcal{B}_{\mat{X}}\left(\mat{P},\mat{Q}\right) = tr\left(\mat{X}^{-1}\mat{P} \mat{X}^{-1}\mat{Q}\right)$ \citep[Appendix A]{boyd2004convex}. We then have:
\begin{align*}
&\max eig(\mathcal{H}) = \max_{\|\mat{P}\|_F=1} \mathcal{B}_{\mat{X}}\left(\mat{P},\mat{P}\right) = \\
&\max_{\|\mat{P}\|_F=1} tr\left(\mat{X}^{-1}\mat{P} \mat{X}^{-1}\mat{P}\right) \leq \\
&\max_{\|\mat{P}\|_F=1} \|\mat{X}^{-1} \mat{P}\|_F^2 \leq \|\mat{X}^{-1}\|_F^2 \leq  \\
& d \|\mat{X}^{-1}\|^2 = \frac{d}{\|\mat{X}\|^2} \leq \frac{d}{\kappa^2},
\end{align*}
where in the last line we denote the spectral norm (maximum singular value) of $\mat{X}$ by $\|\mat{X}\|$. The last inequality is due to the fact that $\mat{X} = \W + \kappa I$, $\W \succ 0$.
We therefore have a bound $M^2_i \leq \frac{\alpha d}{\kappa^2}$.

Finally, the constant $M^3_{i}$ for the Frobenius regularization is immediately seen to be $\beta$.

Collecting all the terms together, we obtain an overall bound on the constant: $M_i \leq M^1_{i} + M^2_{i} + M^3_{i} \leq  M^1_{i} + \frac{\alpha d}{\kappa ^2} + \beta$.
\end{proof}

Let us define a matrix $\Pp \in \R^{d \times d}$ such that $\Pp_{ij} = p_i + p_j$ for $i \ne j$, $\Pp_{ii} = p_i$. $\Pp$ is defined such that $\Pp_{ij}$ is the probability of updating the $(i,j)$ entry of the matrix $\W$ at any given iteration. To show our method converges in a linear rate, we must show that $\tL$, $\Pp$ and the constants $M_i$ satisfy the ``Expected Separable Overapproximation'' assumption presented by \citet{richtarik2013optimal}:

\begin{lemma}[Expected Separable Overapproximation]\label{lem:ESO}
For any symmetric $\Hh \in \R^{d \times d}$ such that $\W + \Hh$ is PSD, let $\Hh^i \in \R^{d \times d}$, $i=1 \ldots d$ be identical to $\Hh$ on the $i$-th row and column, and $0$ elsewhere. Then:
\begin{equation}
\mathbb{E}_{i \sim Mult(p_1, \ldots, p_d)} \left[ \tilde{L}(\W + \Hh^i) \right] \leq 
\tL + \sum_{k,l=1}^d  \grdkl \Hh_{kl} \Pp_{kl} + \frac{1}{2} \sum_{k,l=1}^d   M_k (\Hh_{kl})^2 \Pp_{kl},
\end{equation}
where $i$ is sampled from a multinomial distribution with parameters $(p_1, \ldots , p_d)$.
\end{lemma}

\begin{proof}%[\bf{Proof of Lemma 2}]
\begin{align*}
&\mathbb{E}_{i \sim Mult(p_1, \ldots, p_d)} \left[ \tilde{L}(\W + \Hh^i) \right] =\sum_{i=1}^d p_i \tilde{L}(\W + \Hh^i) \stackrel{(a)}{\leq} \\
& \sum_{i=1}^d p_i \left(\tL + \sum_{k,l=1}^d \grdkl \Hh_{kl}^i + \frac{M_i}{2} \sum_{k,l=1}^d  (\Hh_{kl}^i)^2 \right) \stackrel{(b)}{=} \\
& \tL + \sum_{k,l=1}^d \grdkl \sum_{i=1}^d  p_i \Hh_{kl}^i + \sum_{k,l=1}^d  \sum_{i=1}^d  p_i \frac{M_i}{2} (\Hh_{kl}^i)^2  \stackrel{(c)}{=} \\
& \tL + \sum_{k,l=1}^d \grdkl \Hh_{kl} \Pp_{kl} + \frac{1}{2} \sum_{k,l=1}^d M_k (\Hh_{kl})^2 \Pp_{kl}.
\end{align*}
Inequality (a) is due to Lemma \ref{lem:smooth}. Equality (b) is by changing the order of summation and since the $p_i$ sum to 1. Equality (c) is by a simple counting argument, and the fact that $\Hh^i$ is the restriction of $\Hh$ to its $i$-th row and column. Each off-diagonal element $\Hh_{kl}$ appears twice in the sum over $i$: when $i=k$ and $i=l$. This is accounted for by the elements $\Pp_{kl} = p_k + p_l$.
\end{proof}

\begin{theorem}
Let $\W^t$ be the $t$-th iterate of Algorithm \ref{alg:comet} with objective function $\tL$, sampling each column-row $i$ with probability $p_i$ and using step sizes $\eta_i \leq \frac{1}{M_i}$. Let $\tilde{L}^*$ be the optimal value of $\tL$ on the PSD cone. Let $\beta^* \geq \beta$ be the strong convexity parameter of $\tL$, $M^1 = \max_i M^1_i$, $\Lambda = \max_i \frac{1}{p_i}$, $\rho >0, \epsilon>0$.

If $t > \frac{\Lambda (M^1 + \alpha d (1/\kappa)^2 + \beta)}{\beta^*} log \left( \frac{\tilde{L}(W^0) - \tilde{L}^*}{\epsilon \rho}\right)$ then: $Prob(\tilde{L}(\W^k) - \tilde{L}^* \leq \epsilon) \geq 1-\rho$.
\end{theorem}
\begin{proof}%[\bf{Proof of Theorem 1}]
We show that Algorithm \ref{alg:comet} with objective function $\tL$\footnote{With squared-hinge loss or log loss.}, sampling each column-row $i$ with probability $p_i >0$, and using step sizes $\eta_i \leq \frac{1}{M_i}$, follows Assumption 1 and Assumption 2 of \citet{richtarik2013optimal}. From this the convergence result follows from \citeauthor[Theorem 3]{richtarik2013optimal}, plugging in our bounds regarding the smoothness and strong convexity of $\tL$.

We first note that our algorithm is indeed a special case of the algorithm presented in \citet{richtarik2013optimal}. Specifically, our algorithm assigns probability $p_i > 0 $ to each of the $d$ column-rows of a matrix, and probability $0$ to every other possible choice of coordinates. We update along this block, and the $\log \det$ term acts as a barrier function
assuring us we will stay within the PD cone.

Lemma \ref{lem:ESO} shows our objective is smooth and satisfies Assumption 1 of \citeauthor{richtarik2013optimal}. Assumption 2 of \citeauthor{richtarik2013optimal} is immediately satisfied because of the Frobenius regularization term, ensuring a strong convexity term $\beta^* \geq \beta > 0$. The result follows by considering that the probability $\Pp_{ij}$ of updating coordinate $(i,j)$ obeys $\Pp_{ij} \geq \min_i p_i $ and the values of $M_i$ given in Lemma \ref{lem:smooth}.

\end{proof}





\section*{Appendix E: Results for Reuters CV1 1K features, Caltech256 249 categories and Protein}

\begin{figure}[h]
\centering
\includegraphics[width=7cm]{precision@k_rcv1_4_ig1000}
\caption*{\textit{precision-at-top-k} for REUTERS CV1 dataset with 1000 features}

\end{figure}

\begin{figure}[h]
\centering
\includegraphics[width=7cm]{precision@k_protein}
\caption*{\textit{precision-at-top-k} for Protein (LIBSVM) dataset with 357 features}
\end{figure}

\begin{figure}[h]
\centering
\includegraphics[width=7cm]{precision@k_Caltech256_with_249Categories}
\caption*{\textit{precision-at-top-k} for Caltech256 (249 Categories) dataset with 1000 features}
\end{figure}

\newpage
\small{
\bibliography{comet_supp}
%\bibliographystyle{icml2015}
}


\end{document}



